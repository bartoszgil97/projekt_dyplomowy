\chapter{Wprowadzenie}

Samochód  autonomiczny jest to pojazd sterowany przez system komputerowy, w~oparciu o~dane z~wielu różnorodnych czujników. 
Według klasyfikacji SAE (ang. \textit{Society of Automotive Engineers} -- związek inżynierów zajmujących się motoryzacją) rozróżnia się sześć poziomów autonomizacji:
%TODO 1. de facto to 6 0-5, może je Pan krótko omówić.
%DONE
\begin{itemize}
	\item Poziom 0 oznacza brak jakichkolwiek systemów automatyzujących proces prowadzenia pojazdów,
	\item Poziom 1 realizowany jest przez systemy automatyzujące określone elementy prowadzenia pojazdu. Systemy nie wymieniają się między sobą danymi i odpowiadają na przykład za utrzymanie stałej prędkości, wspomaganie awaryjnego hamowania,
	\item Poziom 2 odpowiada już jeździe półautomatycznej. Na tym poziomie pojazdy mogą autonomicznie jechać prosto, utrzymywać bezpieczny dystans od innych pojazdów i pozostawać na pasie ruchu,
	\item Poziom 3 określa moment kiedy pojazdy może systemy w samochodzie mogą samodzielnie prowadzić pojazd, ale tylko w określonych przypadkach. Jest to jazda półautonomiczna, ale kierowca w każdym monecie może zostać poproszony o przejęcie kontroli nad samochodem,
	\item Poziom 4 to już jazda w pełni autonomiczna. W przez większość czasu pojazd będzie poruszał się autonomicznie i  radził sobie nawet z skomplikowanymi sytuacjami drogowymi. Na tym poziomie zakłada się również komunikację pomiędzy samochodami i będą na przykład informować się o zmianie pasa ruchu,
	\item Poziom 5 charakteryzuje się już pełną autonomiczność pojazdu w każdych warunkach. Osoby przebywające w pojeździe będą już tylko w formie pasażerów.
\end{itemize}
Powyższe poziomy odnoszą się do różnego stopnia zaawansowania systemu ADAS (ang.\textit{Advanced Driving Assistance System} -- zaawansowany system wspomagania kierowcy).
Obecnie produkowane pojazdy umożliwiają nam na jazdę autonomiczną trzeciego stopnia. Co może świadczyć, że pojazdy autonomiczne przestają być tylko wymysłem autorów science fiction.
%TODO Nieprawda. 4 i 5 autonomiczność, 3 częściowoa 0 - brak, 1-2 ADAS.
%DONE

Zadaniem systemów ADAS jest wspomaganie człowieka w czasie jazdy tj. wykrywanie i ostrzeganie o sytuacjach niebezpiecznych dla kierowcy i otoczenia. 
W~sytuacjach krytycznych komputer pokładowy jest w~stanie przejąć kontrolę nad autem i~wykonać manewr zapobiegający wypadkowi. 
Do postrzegania otoczenia wykorzystuje się czujniki takie jak radar, lidar, GPS, kamery, IMU (ang. \textit{inertial measurement unit}) oraz czujniki ultradźwiękowe.



Do podstawowych systemów ADAS zalicza się:
\begin{itemize}
	\item asystent parkowania - w najprostszym wariancie opiera się na czujnikach ultradźwiękowwych rozmieszczone wokół samochodu umożliwiających mierzenie odległości pojazdu od innych przeszkód ułatwiając manewrowanie przy parkowaniu. W najbardziej zaawansowanych umożliwiają już w pełni zautomatyzowane parkowanie,
		%TODO doprecyzować, czy chodzi Panu o parkowania automatyczne, pół-automatyczne, czy same czujniki, co nie jest ADAS.
		%DONE
	\item monitorowanie martwego pola - kamery rozmieszczone bo bokach samochodu zbierają informacje z przestrzeni, których kierowca nie jest w stanie kontrolować wykorzystując z pomocy lusterek,
	\item aktywny tempomat - czujniki odległości znajdujące się z przodu auta mierzą odległość od poprzedzających nas uczestników ruchu drogowego i na tej podstawie zachowuje bezpieczny dystans od innych pojazdów. W sytuacjach niebezpiecznych potrafią również całkowicie zatrzymać pojazd,
	%TODO no to Pan napisał tzw. bzdurę !!! co to jest tempomat

	%TODO swoją drogą to jak ma to działać w przypadku kolizji tylnej
	\item detekcja i rozpoznawanie znaków drogowych - system wykrywa i informuje kierowcę o znakach i~uruchamia sygnał alarmowy w przypadku niedostosowania się do nich,
	\item detekcja samochodów i pieszych - system przy pomocy kamer wykrywa ludzi i pojazdy znajdujące się w najbliższym otoczeniu oraz wyznacza ich przewidywaną ścieżkę ruchu. Jeśli istnieje ryzyko kolizji zostanie uruchomiony alarm. A w sytuacji krytycznej auto zahamuje,
	\item system ostrzegania przed opuszczeniem pasa ruchu (ang. \textit{Lane Departure Warning System}) -- jest to~mechanizm opierający się na detekcji linii drogowych i rozpoznawaniu jezdni. W~sytuacji, w~której samochód zaczyna zmieniać pas ruchu bez uprzednio włączonego odpowiedniego kierunkowskazu system wysyła ostrzeżenie do~kierowcy. Może to być sygnalizacja w postaci wizualnej, dźwiękowej lub wibracji. Wyróżnia się też bardziej zaawansowane wersje oprogramowania, w~których system przejmuje kontrolę nad układem kierowniczym i~nie pozwala na~zmianę pasa ruchu przez pojazd lub kieruje nim na środek pasa. Tego typu systemy zostały zaprojektowane w~celu zminimalizowania liczby wypadków drogowych wynikających z~błędów kierowców powodowanych między innymi przez zmęczenie bądź utratę koncentracji.
\end{itemize}

%TODO Może Pan dopisać, że z zaawansowanych wersji powyższych systemów można stworzyć pojazd autonomiczny.
% "Powyższe poziomy odnoszą się do różnego stopnia zaawansowania systemu ADAS (ang.\textit{Advanced Driving Assistance System} -- zaawansowany system wspomagania kierowcy)." To zdanie nie zawiera tej informacji?

W ostatnim czasie można zauważyć spory rozwój w dziedzinie pojazdów bezzałogowych. Jest to~spowodowane widocznym rozwojem w dziedzinie sensoryki. oraz jednostek obliczeniowych w dziedzinie CPU (ang. \textit{Central Processing Units}) i GPGPU (ang. \textit{General Purpose computing on Graphics Processing Units}). Dodatkowo kolejny firmy motoryzacyjne starają się osiągać coraz to lepsze rozwiązania w celu wyróżnienia się na tle konkurencji i zdobycia większego rozgłosu. Na horyzoncie już widać pierwsze przymiarki do oficjalnego startu sezonu wyścigów samochodów autonomicznych Roborace \cite{roborace}. Przez długi czas wyścigi samochodowe były poligonem doświadczalnym dla nowych technologii, które następnie mogły trafić do samochodów drogowych. Aktualnie większość elektronicznych wspomagaczy kierowcy zostało zakazanych w Formule 1 i wpływ wyścigów uległ osłabieniu. Ale skoro w przyszłości mamy poruszać się pojazdami autonomicznymi, Roborace może pomóc w opracowywaniu nowej technologii takich pojazdów.

Jednym z możliwych rozwiązań efektywnego rozwoju oprogramowania są układy Zynq SoC. Dzięki możliwości ich reprogramowania są wygodnym zamiennikiem układów ASIC (ang. \textit{Application-Specific Integrated Circuit} - układ scalony specyficzny dla aplikacji) \cite{ASIC_zynq}. Rekonfigurowalność układu zapewnia szybszy i tańszy rozwój oprogramowania oraz pozwala na naprawę błędów w oprogramowaniu bez konieczności tworzenia nowego układu.
Firma Xilinx oferuje specjalistyczne układy z myślą o systemach ADAS. Urządzenia takie jak XA Zynq™-7000 SoCs idealnie nadają się do wysokich wymagań obliczeniowych zaawansowanych systemów wspomagania kierowcy (ADAS) \cite{xilinx}.


\section{Cele pracy}
W niniejszej pracy podjęto temat detekcji jezdni dla potrzeb pojazdów autonomicznych.  
Celem pracy jest napisanie modelu programowego w~języku Matlab algorytmów przetwarzających nagrany obraz. 
Rezultatem powinien być film, na który naniesione są krzywe reprezentujące detekcje linii drogowych.
Kolejną częścią pracy jest przeprowadzenie implementacji sprzętowej algorytmów na układzie Zynq SoC(ang. \textit{System On Chip}) przy użyciu języka opisu sprzętu Verilog. 
\section{Układ pracy}
W rozdziale drugim przedstawiono przegląd metod przetwarzania obrazu wykorzystywanych w systemach ADAS. 
W~kolejnym rozdziale omówiono platformę Zybo Zynq SoC. 
Rozdział czwarty odnosi się do implementacji programowej. Natomiast w rozdziale piątym poruszona zostaje kwesta implementacji sprzętowej. Szósty rozdział zawiera opis przeprowadzonych testów. Pracę kończy podsumowanie.