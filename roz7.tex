\chapter{Podsumowanie}

\section{Wnioski}
W ramach pracy inżynierskiej udało się zaimplementować model programowy w języku Matlab. 
Algorytm charakteryzował się jednak zauważalnym czasem obliczeń. Implementacje sprzętowe binaryzacji oraz wyznaczania obszaru zainteresowanie ROI udało się przetestować symulacyjnie. Porównanie wyników z implementacją programową świadczy o tym, że algorytmy zostały poprawnie zaimplementowane.
Symulacja możliwa była do przeprowadzanie dla obrazów o wymiarach 348x128. Dlatego też nie udało się przeprowadzić symulacji filtru LMPS. Znaczne zmniejszenie i zniekształcenie względem oryginalnego obrazu powoduje, że filtr nie daje rzetelnych rezultatów. Zamieszczone testy przeprowadzono na modelu programowym.
Przedstawionego podejścia nie udało się zaimplementować na platformie Zybo Zynq SoC ze względu na zdarzenia losowe.

Na podstawie przeprowadzonych testów można stwierdzić, że detekcja linii drogowych za pomocą przedstawionego podejścia jest łatwiejsza dla drogi prostej.
Światło odbijające się od drogi wprowadzało znaczące utrudnienia w~poprawnym wykrywaniu obu linii pasów ruchu drogowego. Ale pomimo refleksów udawało się względnie wykryć część pasów.
Dodatkowo sporym utrudnieniem może być zacienienie jednego z pasów przy bardzo słonecznej pogodzie.
W takim przypadku pas staje się praktycznie nie widoczny i odfiltrowany w procesie binaryzacji.
Na filmie z~autostrady widać, że podczas manewru zmiany pasa ruchu, program nie jest w stanie wykrywać obu linii, gdyż statyczne roi odfiltrowuje jeden z pasów. Ale dalej możemy posiłkować się detekcją pojedynczego pasa.
Wprowadzenie sytuacji niestandardowej jak żółte pasy ruchu drogowego niezbyt wpłynęła na detekcją linii drogowych.
Najbardziej znaczącym czynnikiem na jakość detekcji jezdni jest jakość oświetlenia, w przypadku mocnego i~niejednolitego algorytm nie radził sobie najlepiej.