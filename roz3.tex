\chapter{Opis platformy sprzętowej}

\section{Specyfikacja sprzętowa platformy Zybo}

\begin{figure}[h]
	\centering
	\includegraphics[scale=2]{zybo_img.png}
	\caption{Platforma Zybo z~układem Zynq SoC. Źródło: \cite{zybo_img}}
	\label{fig:zybo_img}
\end{figure}

Do implementacji sprzętowej została wykorzystana karta Zybo firmy Digilent \ref{fig:zybo_img} wyposażona w~układ Zynq SoC (ang. \textit{System on Chip}). 
W~dokumencie \cite{zybo_description} znajduje się opis budowy układu. 
Jest on określany mianem "heterogeniczny", ponieważ zawiera dwa rodzaje zasobów sprzętowych: dwurdzeniowy procesor ARM Cortex-A9 oraz układ FPGA(ang.\textit{Field Programmable Gate Array} -- bezpośrednio programowalna macierz bramek), czyli logikę reprogramowalną.

\subsection{Logika reprogramowalna}



Część reprogramowalna układu Zynq na karcie Zybo jest oparta o~logikę serii  Artix-7 firmy Xilinx. 
Podstawowym elementem z~którego zbudowane jest FPGA, to blok CLB (ang. \textit{Configurable Logic Block}). 
Składa się on z~dwóch Slice'ów połączonych z~matrycą przełączeń (ang. \textit{Switch Matrix}). 
W~układach Zynq występują dwa rodzaje elementów Slice, są to SliceL i SliceM. 
Slice typu M składa się z:
\begin{itemize}
	\item generatora funkcyjnego (4 sztuki) -- został on zrealizowany przy pomocy układów LUT (ang. \textit{Look-Up Table}). Posiada 6 wejść i~2 wyjścia. Może także zostać skonfigurowany jako synchroniczna pamięć RAM lub 32 bitowy rejestr przesuwny wykorzystywany w~liniach opóźniających,
	\item przerzutnika typu D (FF -- ang. \textit{Flip-Flop}) -- slice zawiera 8 sztuk, przy czym 4 mogą zostać skonfigurowane jako zatrzask (ang. \textit{latch}),
	\item szybkiej logiki przeniesienia,
	\item multiplekserów.
\end{itemize}

Do pozostałych zasobów dostępnych w układach FPGA serii Artix-7 należą:
\begin{itemize}
	\item CMT (ang. \textit{Clock Managment Tiles}) -- bloki umożliwiające zarządzanie sygnałem zegarowym, generowanie różnych częstotliwości zegara, równomierną propagację sygnału, tłumienie zjawiska zakłócenia fazy zegara,
	\item Block RAM (BRAM) -- blokowa dwuportowa pamięć RAM o~rozmiarze 36Kb (na blok). Może zostać skonfigurowana jako moduł FIFO (ang. \textit{First In First Out}), 
	\item GTP/GTX Transceivers -- moduły umożliwiające transmisję szeregową z~prędkością do 12,5 Gb/s (GTX) i~6,25 (GTP),
	\item DSP48A1 -- moduł zawierający mnożarkę 25x18 bitów oraz akumulator 48 bitowy. Liczba modułów zależy od rozmiaru układu i~zawiera się w przedziale od 66 do 2020,
	\item Select I/O -- banki zasobów wejścia/wyjścia, których liczba zawiera się w~przedziale od 100 do 400 końcówek podłączonych do części FPGA.
\end{itemize}

\subsection{System procesorowy}
ARM Cortex-A9 MPCore jest 32 bitowym procesorem firmy ARM Holdings z~zaimplementowaną archtekturą ARMv7-A. 
Zawiera od 1 do 4 rdzeni. 
Płytka Zybo Zynq SoC jest wyposażona w~wersje procesora dwurdzeniowego.
Rozkazy procesorów ARM są tak skonstruowane, aby wykonywały jedną określoną operację w~jednym cyklu maszynowym.
Kluczowymi cechami rdzenia Cortex-A9 są \cite{armCortex}:

\begin{itemize}
	\item NEON SIMD (ang. \textit{single instruction, multiple data} -- pojedyncza instrukcja, wiele danych) opcjonalne rozszerzenie zestawu instrukcji do 16 operacji na instrukcję,
	\item rozszerzenia zabezpieczeń TrustZone,
	\item jednostka zmiennoprzecinkowa VFPv3 dwukrotnie przewyższająca wydajność swojego poprzednika ARM FPUs,
	\item kodowanie zestawu instrukcji Thumb-2 zmniejsza rozmiar programów, co poprawia wydajność,
	\item program Trace Macrocell i~CoreSight Design Kit do nieinwazyjnego śledzenia wykonywania instrukcji,
	\item przetwarzanie wielordzeniowe,
	\item kontroler pamięci podręcznej L2 (0-4 MB),
	\item kontroler pamięci statycznej, dynamicznej i bezpośredniej.
\end{itemize}
